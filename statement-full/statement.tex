\documentclass[12pt]{article}
\usepackage[hmargin=0.8in,vmargin=0.8in]{geometry}\usepackage[T1]{fontenc}
\usepackage{graphicx}
\usepackage{url}

\setlength{\topskip}{0mm}
\def\parskipval{0em}\def\parindentval{1em}
\setlength{\parskip}{\parskipval}\setlength{\parindent}{\parindentval}
\usepackage{enumitem}\setlist{nolistsep}

\usepackage{natbib}
\setlength{\bibsep}{0.0pt}

\title{Runtime analysis of mobile devices for malware detection. \\
  {\large Research proposal.}}
\author{Brandon Amos}
\date{\today}

\begin{document}
\maketitle

\section{Introduction.}
Mobile devices and technology are exploding,
where the number of mobile devices
is expected to exceed the number of people on Earth by
2014\footnote{\url{http://blogs.cisco.com/sp/the-future-of-monetizing-mobility/}}.
% and wireless carriers globally
% obtain \$1.3 trillion in revenue each year.
%outside of computer science
%Major industries
%utilize mobile technologies,
%where the U.S. military have over 250,000 mobile devices in
%use, %\footnote{\url{http://www.informationweek.com/government/mobile/dod-pushes-militarys-mobile-strategy-for/240010603}}
Mobile security has become a serious concern due to the sensitive
information mobile devices contain and the critical systems mobile
devices control. %, such as GPS locations of soldiers.
% Compromising mobile devices grants hackers access
% and control to critical information and cyber-physical systems,
% and compromising mobile devices raises major security concerns.
In {\it the military}, the GPS locations and identities of
deployed military members can be obtained.
In {\it manufacturing}, computer-aided manufacturing tools can be
controlled with a mobile device,
allowing an attacker to silently alter the manufacturing process,
as I have studied \cite{turner2013bad}.
In {\it healthcare}, patients can rely on glucose monitors on
mobile devices, allowing an attacker to send false reports and harm patients.
\textit{Low malware detection rates are an \textbf{open research problem.}}
A study of thousands of real Android malware samples in 2011 show
that industry antimalware software detect
79.6\% of the malware in the best case and
20.2\% in the worst case \cite{zhou2012dissecting}.

\newpage
\bibliographystyle{abbrv}
\bibliography{references}

\end{document}
